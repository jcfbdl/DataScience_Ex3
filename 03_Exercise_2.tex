\chapter*{Exercise 2}

In order to contain the weight instability, we can amend the covariance matrix in order to “artificially” always make it useful and stable i.e. invertible. One method allowing us to end up with an invertible matrix is known as matrix shrinkage. \par\smallskip
Shrinkage works by computing a convex combination of a positively defined matrix with a semi-positively defined matrix, the result is a positively defined matrix which, by nature, is invertible. Furthermore, provided the weighting of the positively defined matrix is sufficient, the invert of combination not only exists, but is also stable. \par\smallskip
We use the following regularisation with $\lambda=0.2$ given to shrink the covariance matrix:
\begin{equation*}
\Sigma_s = \lambda F + (1-\lambda)\Sigma
\end{equation*}

Here $\Sigma$ is the semi-positively defined matrix whilst $F$ is going to be the positively defined one, namely:

\begin{itemize}
    \item $F_1=I_n$ The Identity Matrix
    \item $F_2=F_{\overline{\rho}}$ The Constant Correlation Matrix, first used by Ledoit and Wolf. Here we substitute the correlation coefficients in the covariance matrix between each pair of stocks with the average of the correlation coefficients.
\end{itemize}

\section*{Question 1}

In the first case we use the identity matrix $I_n$ and $\lambda=0.2$. We transform our estimate of $\Sigma$ and compute the inverse in order to solve the optimal weights. \\
We determine the optimal weights using both $\mu$ and $\Sigma$ estimates from the same sub-sample but using the shrunk covariance matrix $\Sigma_S$ to compute the optimal weights. In Figure \ref{fig2}, the left box-plot shows the resulting weight distribution for the first 20 stocks. The right box-plot is the plot from Figure \ref{fig1} also obtained using only the sub-sample but without shrinkage. \\
Clearly, using a shrunk covariance matrix drastically improves the stability of the weights. However, this \emph{does not guarantee the optimality of the weights themselves}, we would have to test the performance of these new weights and compare said performance to that of the optimal portfolio derived from the full-sample.

Figure \ref{fig2} compares  
\begin{itemize}
\item The weight distribution estimated from the sub-sample (Figure \ref{fig1} "mu and sigma updated"),
\item our estimates of sigma based on the entire dataset and mu based on the subsample (second boxplot "mu updated"), and
\item the estimates based on the subsample but using the first shrunk covariance matrix, where $F=I$ ("regularized F=I"). 
\end{itemize}

\begin{figure}[H]
%\includegraphics[width=15cm]{DS_Ex3_Q2A.eps}
\caption{Weight stability with shrinkage $F=I$}
\label{fig2}
\end{figure}

Note the different scale on the y-axis. We can see that...

 
Figure \ref{fig3} differs from Figure \ref{fig2} only in the third diagram. Here we compare our previous results with the estimates using the constant correlation matrix.
\begin{figure}[H]
%\includegraphics[width=15cm]{DS_Ex3_Q2B.eps}
\caption{Weight stability with shrinkage using the constant correlation matrix}
\label{fig3}
\end{figure}

We can see that...
