\section*{Exercise 2}

We need to show the stability of the weights on the optimal portfolio with shrunk covariance matrices. We use the regularization 
\begin{equation*}
\Sigma_s = \lambda F + (1-\lambda)\Sigma
\end{equation*}
with given $\lambda=0.2$ and with the following two substitutions:
\begin{enumerate}
\item $F=I$, the identity matrix 
\item $F =$ the constant correlation matrix, first used by Ledoit and Wolf. Here we substitute the correlation coefficient between each pair of stocks with the average of the correlation coefficients.  
\end{enumerate} 


In order to compute optimal portfolio weights, the covariance matrix needs to be invertible and needs to be estimated as precisely as possible. 
Shrinkage of the covariance matrix is a way to overcome the problems of its noisy estimation and non-invertibility.
The regularization ensures that the shrunk covariance matrix will be positive definite and thus invertible. We create a convex combination of the estimated covariance matrix, which is potentially positive semi-definite, and a structural matrix $F$, which is designed to be positive definite.

Figure \ref{fig2} compares  
\begin{itemize}
\item our estimates based on the subsample (first boxplot "mu and sigma updated"),
\item our estimates of sigma based on the entire dataset and mu based on the subsample (second boxplot "mu updated"), and
\item the estimates based on the subsample but using the first shrunk covariance matrix, where $F=I$ ("regularized F=I"). 
\end{itemize}

\begin{figure}[H]
%\includegraphics[width=15cm]{DS_Ex3_Q2A.eps}
\caption{Weight stability with shrinkage $F=I$}
\label{fig2}
\end{figure}

Note the different scale on the y-axis. We can see that...

 
Figure \ref{fig3} differs from Figure \ref{fig2} only in the third diagram. Here we compare our previous results with the estimates using the constant correlation matrix.
\begin{figure}[H]
%\includegraphics[width=15cm]{DS_Ex3_Q2B.eps}
\caption{Weight stability with shrinkage using the constant correlation matrix}
\label{fig3}
\end{figure}

We can see that...
